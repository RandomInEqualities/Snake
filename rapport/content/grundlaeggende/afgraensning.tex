\section{Afgrænsning}
Simpel Snake er blevet opbygget af en torrus, som er selve banen, en slange, der styres af piletasterne, og noget mad, der kan gøre slangen længere.

\paragraph{Banen}
Banen er opbygget som en todimensionel torus, der har den funktion, at hvis man bevæger sig ud over banens kanter, fortsætter slangen på den modsatte side.
Banen er opdelt i felter af typen Field, som angiver positionerne rundt på banen. Dette bruges i måden slangen bliver konstrueret på.

\paragraph{Slangen}
Slangen består af en ArrayList af typen Field, der angiver hver position for slange-kroppen og slange-hovedet. Det er muligt gennem input fra tastaturet at bevæge sig frit på banen i de fire retninger i et almindeligt koordinatsytem: op og ned ad y-aksen og hen ad x-aksen i begge retninger - dog ikke modsatrettet slangens bevægelsesretning. For at visualisere slangen på torussen, bliver hvert felt på banen, som har Field-positionerne i slangens ArrayList, farvet sort. På den måde kan slangen bevæge sig, ved at rykke ArrayListens elementer frem og lave et nyt hoved i den retning piletasterne har registreret.
Ved spillets start består slangen af to felter (et hoved og en hale), der er placeret i banens centrum med hovedet rettet mod venstre. 

\paragraph{Mad}
Det skal derudover være muligt at øge slangens længde ved at bevæge hovedet over et felt med mad. Når slangen spiser mad, bliver ArrayListen ikke rykket, men maden bliver til slangens nye hoved. 

\paragraph{Spillet slutter}
Spillet slutter når, spilleren bevæger sig ind i et felt udfyldt af slangen, altså spiser sig selv. I så fald, har spilleren tabt. Man har vundet spillet, når spilleren har fyldt hele banen ud med slangens krop, så der ikke er flere felter tilbage at placere mad på. 