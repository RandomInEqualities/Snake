\section{Afgrænsning}
Simple Snake er opbygget af en bane inddelt i felter og en slange dannet af felter. Spilleren styrer slangen, der ved spillets start består af to felter: et hoved og en hale. Spilleren skal ved hjælp af piletasterne styre slangen ind i et farvet felt på banen, der skal forestille at være mad. Det farvede felt forsvinder derefter, og slangen bliver et felt længere og får et point.

\paragraph{Banen}
Banen er opbygget som en todimensionel $n\times m$ torus, ved at bevægelse ud af den ene side, er en bevægelse ind fra den modsatte side. Banen er opdelt i felter (rækker, kolonner) som angiver positionerne på banen. Banen skal tilpasse sig vinduets størrelse som er justerbart, og det skal uden problemer være let at ændre størrelsen i programmets kode.

\paragraph{Slangen}
Slangen består af et antal sammenbundne felter. Den har et hoved og en hale der er de yderste felter af slangen. Slangens hoved har bevægelsesretningerne op, ned, højre og venstre. Gennem input fra tasteturet skal det være muligt at bevæge slangen rundt på banen. Dog må den ikke kunne bevæge sig i modsat retning i forhold til dens hoveds retning. På skærmen skal slangen visualiseres ved nogle firkanter som repræsenterer slangens felter.

\paragraph{Mad}
På banen skal der være et felt med mad, som slangen kan spise. Når dette sker bliver slangen et felt længere, maden forsvinder og dukker op i et andet felt på banen. Maden skal placeres tilfældigt på et af banens tomme felter. Når maden bliver spist, fås et point som opdateres i et tekstfelt.

\paragraph{Kollision}
Spillet slutter, når slangen bevæger sig ind i et felt udfyldt af slangen selv medmindre det er halen, som også flyttes, når resten af slangen flyttes. Kolliderer slangens hoved med dens egen krop, har spilleren tabt. Fylder spilleren hele banen ud med slangens krop, har han derimod vundet. Når spilleren enten taber eller vinder, skal spillet fryse, ved at tastatur-input ikke længere kan bevæge slangen.