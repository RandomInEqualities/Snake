\section{Afgrænsning}
Simpel Snake er opbygget af en torrus, som er selve banen, en slange, der styres af piletasterne, og noget mad, der kan gøre slangen længere. Selve afgrænsning af denne del er defineret af projektoplægget, så det vil være bedst at læse dette for en uddybbende afgrænsning. Vi laver et resume over oplægget.

\paragraph{Banen}
Banen er opbygget som en todimensionel $n\times m$ torus. Hvis man bevæger sig ud over banens kanter, fortsætter man på den modsatte side. Banen er opdelt i felter $(x,y)$ som angiver positionerne på banen. 

\paragraph{Slangen}
Slangen består af et antal sammenbundne punkter. Den har et hoved og en hale. Slangens hoved har en bevægelsesretning $op$, $ned$, $højre$ eller $venstre$. Gennem input fra tastetured skal det være muligt at bevæge slangen. På skærmen skal slangen visualiseres via nogle firkanter som representere slangens punkter.

\paragraph{Mad}
På banen skal der være et felt med mad, som slangen kan spise. Når dette sker bliver slangen et punkt større. Efter at slangen har spist maden, skal der sættes et nyt mad punkt på banen. Dette punkt skal blive fordelt uniformt tilfældigt på banen.

\paragraph{Kollision}
Spillet slutter når, slangen bevæger sig ind i et felt udfyldt af slangen selv. I så fald, har spilleren tabt. Man har vundet spillet, når spilleren har fyldt hele banen ud med slangens krop. Bemærk at slangen godt kan bevæge sig ind i sit hale felt, da halen også flytter sig i det samme træk.