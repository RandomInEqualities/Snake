\section{Afgrænsning}
I Simpel Snake består slangen af to felter (et hoved og en hale), der er placeret i banens centrum med hovedet rettet mod venstre. Det er muligt gennem input fra tastaturet at bevæge sig frit på banen i de fire retninger i et almindeligt koordinatsytem: op og ned ad y-aksen og hen ad x-aksen i begge retninger - dog ikke modsatrettet slangens bevægelsesretning. Bevæger man sig ud over banens kanter, fortsætter slangen på den modsatte side. Det skal derudover være muligt at øge slangens længde ved at bevæge sig over et felt med et æble. Spillet slutter, når spilleren har fyldt hele banen ud med slangens krop og dermed har vundet spillet, eller ved at spilleren bevæger sig ind i et felt udfyldt af slangen, der ikke er halestykket, da halestykket ved slangens bevægelse også rykker videre til et nyt felt. I så fald, har spilleren tabt.