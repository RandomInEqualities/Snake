\section{Evaluering}
-To eller en Arraylist -> linklist ?
-Point -> Field
-Int -> Enum (direction)
-width+height <-> kvadratisk variabel ?
-Banens opbygning: double-array -> positions-placering
-Optimering af runtime i “food”-klassen (dobbelt for-loop eller random placering)
-brug af observer - får model-view-controller til at gå op
-Flere opgaver er givet til “game”-klassen frem for de andre klasser

\subsection{Arbejdsproces}
Formålet med projektet var at starte med at lave en simpel udgave af snake, og derefter tilføje flere funktioner for at lave en mere avanceret version. Dette gør det ideelt at tilføje én funktion ad gangen, frem for at planlægge alle funktioner på en gang, og tilføje dem samtidig. Resultatet bliver et, til at starte med, simpelt men fungerende program, hvorefter yderligere funktioner kan tilføjes. Programmet er altså udviklet iterativt, hvorved der opstår flere fungerende versioner af spillet, men med forskellige funktioner. Dette gør det muligt at tilpasse programmet, hvis der opstår nye idéer eller krav undervejs. 

Den iterative tilgang gør det muligt at have en "cyklus" for udviklingen af programmet. Først bestemmes det, der skal tilføjes til programmet. Derefter fordeles opgaverne blandt gruppens medlemmer. Gruppemedlemmet afgør selv, hvordan en funktion skal designes og implementeres, men sikrer at implementationen er kompatibel med alle nuværende funktioner, og ikke vil hindre fremtidige tilføjelser i at blive tilføjet. Eventuelle justeringer til programmet laves for at undgå fejl med nye funktioner, hvorefter "cyklussen" starter forfra ved idéfasen.
