\section{Design}
Snake-spillet er lavet efter et \textit{Model-View-Controller}-design (MVC), hvorved selve spillet, styringen af spillet og den visuelle repræsentation af spillet holdes adskilt i tre dele. På denne måde interagerer brugeren kun med den del af programmet, der er dedikeret til styring. Styringen manipulerer programmets tilstand i \textit{model}-koden, som visualiseres i \textit{view}-koden. Dette betyder derfor også, at alle funktioner, der påvirker programmets tilstand, skal holdes i \textit{model}-koden. \textit{Control} modtager kun input fra brugeren, og sender dette videre til \textit{View} og/eller \textit{Model}. \textit{View} modtager input fra \textit{Control} og sender ændringerne i \textit{View} videre til \textit{Model}. Det er derefter muligt gennem en observer at "observere" ændringerne i \textit{Game}, som bliver opdateret i \textit{View}.

Simpel Snake er designet, så spillets funktioner ligger i klassen \textit{Game} i model-pakken. \textit{Game} får da spillets objekter fra andre klasser i \textit{model}-pakken, f.eks. Level (banen) og Snake (slangen). Spillets tilstand ændres, når der modtages input fra control-klasserne, som bestemmer hvornår og hvordan slangen bevæger sig. I \textit{BoardPanel}-klassen er en observer, som notificeres hver gang banens tilstand ændres. Når dette sker, eller når spillet startes, tegnes banen vha. klasserne \textit{View} og \textit{BoardPanel}. Disse klasser modtager information fra model-klasserne, f.eks. \textit{Snake}, til at bestemme hvordan spillet tegnes. \textit{BoardPanel} gentegner hele spillet hver gang spillet opdateres. Først tegnes selve banen, derefter slangen og til sidst æblet. 

For at starte spillet bruges klassen \textit{Driver}, som opretter et nyt \textit{Game}-, \textit{View} og \textit{Control}-objekt.