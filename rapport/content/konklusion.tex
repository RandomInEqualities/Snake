\chapter{Konklusion}

Formålet med projektet var at starte med at lave en simpel udgave af snake, der ville være let at udvide til en avanceret version ved at tilføje flere funktioner. Dette gør det ideelt at tilføje én funktion ad gangen, frem for at planlægge alle funktioner på en gang og tilføje dem samtidig. Resultatet bliver et, til at starte med, simpelt men fungerende program, hvorefter yderligere funktioner kan tilføjes. Programmet er altså udviklet iterativt, hvorved der opstår flere fungerende versioner af spillet, men med forskellige funktioner. Dette gør det muligt at tilpasse programmet, hvis der opstår nye idéer eller krav undervejs, og altså lettere at opgradere Simple Snake til Advanced Snake.
\newline

I vores version af Snake, valgte vi at fokusere på en ren og fleksibel brugergrænseflade og en struktureret opbygning af programmet, der ville gøre det let at foretage ændringer og implementere nye funktioner. Dette blev gjort med \textit{Model-View-Controller}-designet, som holdte forskellige klasser adskilt. Derudover gjorde designet det også let at finde rundt i klasserne, når fejl skulle rettes, eller algoritmer skulle ændres.
\newline

På trods af meget importeret materiale og mange klasser, forsøgte vi dog også at mindske programmets størrelse, så meget som muligt, for at kunne overskue det. Dette gjorde vi ved at lade nogle klasser bruge metoder fra andre klasser, f.eks. ved tegning af baggrunden. Forskellige klasser, der har meget tilfælles, har vi også ladet nedarve fra en fælles superklasse, der gør genbrug af kode simpel, og underklasserne kortere. Dette ses f.eks. ved \textit{HeaderBasePanel}-klassen der er top-panelet, når man ikke er inde i spil-scenerne. Denne bliver nedarvet af \textit{HeaderSinglepalyerPanel} og \textit{HeaderMultiplayerPanel}, som ses inde i spil-scenerne. Disse genbruger superklassens udseende og knapper, samtidig med at tilføje deres score-tekstfelter. Også \textit{OptionsSingleplayerListener} og \textit{OptionsMultiplayerListener} nedarver fra superklassen \textit{OptionsListener}, der dog er en abstrakt klasse, idet den aldrig skal kunne oprettes for sig selv - ligesom den abstrakte klasse \textit{Game}.
\newline

Evalueringen af Simple Snake kunne bruges til videre udvikling af spillet til Advanced Snake, hvor f.eks. skalering af spillet spillede en stor rolle i begge versioner. I den første version bevarede felterne ikke deres størrelsesforhold, mens dette var bevaret i den nye version, hvor noget af vinduet dog derimod ikke blev fyldt helt ud af banen. Nogle af funktionerne i Simple Snake er også bevaret i Advanced Snake, f.eks. brugen af ArrayList, oprettelsen af æblet og 'move'-metoden, der dog er ændret en smule, men har beholdt den oprindelige metodes fremgangsmåde, da den viste sig at være effektiv.

Med implementeringen af flere scener, paneler og ActionListeners, der gjorde musseklik muligt, var det også vigtigt at have kontrol over hvilket panel der fik fokus og hvornår panelet mistede fokus, idet selve spil-scenen kunne miste fokus, når spillet blev startet og derved gøre det umuligt for spilleren at styre slangen. Med nye implementeringer er det altså generelt vigtigt straks at løse de problemer der følger, da det ellers kan give fejl, der er svære at finde og gennemskue, i det ellers tidligere fungerende spil.
