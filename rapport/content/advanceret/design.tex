
\section{Design}
I \textit{Driver}-klassen, hvori main-metoden for programmet ligger, oprettes to objekter fra de to hovedklasser \textit{View} og \textit{Control}, der hver især står for henholdsvis \textit{View} og \textit{Control} i \textit{Model-View-Controller}-designet. Klassen \textit{Game}, der står for \textit{Model}-delen oprettes først i hovedmenuens control-klasse \textit{ControlMenu}, da denne derved får lettere adgang til spillet, når spilleren trykker på menuens knapper. 

I JFrame'n \textit{View} er spillets forskellige komponenter opdelt i hver deres klasse. Disse view-klasser er alle JPanels som tilføjes til vinduet og skiftes ud alt efter spillets tilstand. Er spilleren f.eks. i hovedmenuen, bruges \textit{ViewMenu}-klassen til at tegne denne. Er spilleren i et multiplayer-spil, bruges \textit{ViewBoardMultiplayer} og \textit{ViewHeaderMultiplayer}. På denne måde har hver scene i spillet en tilhørende klasse der forlænger et JPanel. De fleste af view-klasserne oprettes alle med \textit{View} som parameter, og kan derfor give og bruge hinandens funktioner, idet \textit{View}-klassen har getter-metoder til de view-klasser, som oprettes i \textit{View}. View-klasserne kan derfor genbruge nødvendige metoder, f.eks. metoden til at tegne baggrunden op, som er defineret i \textit{ViewMenu}-.klassen. Spilleren navigerer vha. JButtons eller tastatur-input der modtages i control-klasserne.
View-klasserne består af \textit{Colors}, \textit{Images}, \textit{ViewAudio}, \textit{ViewBoard}, \textit{ViewBoardMultiplayer}, \textit{ViewBoardSingleplayer}, \textit{ViewControls}, \textit{ViewHeader}, \textit{ViewHeaderSingleplayer}, \textit{ViewHeaderMultiplayer}, \textit{ViewMenu}, \textit{ViewOptions}, \textit{ViewOptionsSingleplayer} og \textit{ViewOptionsMultiplayer}. [SE KLASSEDIAGRAM FOR RELATIONER....]
Al grafik, som ikke er fra \textit{Swing}-biblioteket, er importeret i klassen \textit{Images}, der gør det muligt for enhver view-klasse at få fat i alle billeder. På samme måde er selv-definerede farver oprettet i klassen \textit{Colors}.

\textit{Control}-klassen er konstrueret med \textit{View} som parameter, og virker derfor alle steder i spillet. \textit{Control}-klassen har funktioner som \textit{mute} og \textit{return to menu}, som skal være fungerende uanset hvor i spillet spilleren er, og dermed uanset hvilke paneler der er vist. I samme pakke ligger andre control-klasser, som alle hører til en bestemt scene. Dette sikrer, at der kun bliver givet de styringfunktioner, som er nødvendige i den nuværende scene.

I \textit{Model}-pakken findes den abstrakte klasse \textit{Game}, som klasserne \textit{GameSingleplayer} og \textit{GameMultiplayer} implementerer. Klasserne har handlinger (MOVE, EAT, LOSE, WIN, START, PAUSE, RESUME) og spiltilstande (isStarted, isRunning, isPaused, isEnded) tilfælles, men har forskellige mål, der derfor er defineret i hver deres klasse, mens de fælles funktioner ligger i superklassen.