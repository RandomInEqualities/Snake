\section{Design}
Den avancerede versions design er også baseret på Model-View-Controller-conceptet, og indeholder derfor pakkerne, \textit{model}, \textit{view} og \textit{control}. Klasserne i control-pakken er uafhængige af klasserne i de andre pakker, men oprettes og tildeles i view-klasser, der giver kontrol over spillets progression og brugergrænsefladens udseende.

I \textit{Driver}-klassen, hvori main-metoden for programmet ligger, oprettes kun et \textit{ViewFrame}-objekt, der nedarver fra JFrame, og viser spillets vindue. I denne klasses constructor oprettes alle andre relevante klasser, som findes i view-pakken [REFERENCE TIL KLASSEDIAGRAM]. Disse nedarver fra JPanel, og tilføjes og fjernes fra JFramen afhængig af hvor i programmet spilleren befinder sig. Er spilleren f.eks. i hovedmenuen, bruges \textit{MenuPanel}-panelet, mens \textit{HeaderMultiplayerPanel} og \textit{BoardMultiplayerPanel} tilføjes, hvis spilleren er i multiplayer-delen af spillet. På denne måde har hver scene i spillet en eller to tilhørende klasser der nedarver fra JPanel. 

Spilleren navigerer vha. JButtons eller tastatur-input der modtages i control-klasserne. Control-klassen \textit{ViewFrameListener} oprettes ligeledes i \textit{ViewFrame}s constructor, hvorefter den som KeyListener tilføjes til alle panelerne. Denne klasses funktioner er globale, da den står får `mute'- og `return to menu'-funktionerne, der konstant skal være tilgængelige for spilleren, uafhængig af de viste paneler.

Hvert panel opretter også i deres egne constructors deres tilhørende Listener fra control-pakken. Disse Listeners står for kontrollen, der er unik for deres panel. Alle panel-klasserne oprettes med \textit{ViewFrame} som parameter, der derefter igen bruges som parameter, når panel-klassens control-klasse oprettes. Control-klassen kan derefter bruge \textit{ViewFrame}, til at skifte paneler ud, når knapper eller taster trykkes. Specielt for \textit{MenuListener}, der hører til hovedmenu-panelet, oprettes i constructoren også et \textit{GameSingleplayer}- og et \textit{GameMultiplayer}-objekt, da disse skal bruges, når der klikkes på `Singleplayer'- eller `Multiplayer'-knapperne. 

\textit{GameSingleplayer} og \textit{GameMultiplayer} er underklasser til \textit{Game}-klassen, og står for spil-delens oprettelse ved at bruge de andre objekt-klasser i model-pakken: \textit{Board}, \textit{Food} og \textit{Snake}. Derudover findes der i model-pakken hjælpeklasserne: \textit{Field}, \textit{Direction}, \textit{Event} og \textit{Player}. \textit{Game} extends Observable-klassen, der gør det muligt for view-klasserne at blive notificeret, når der foretages ændringer i spillet, og følgelig tilpasse sig ændringerne i brugergrænsefladen.

Al grafik, som ikke er fra \textit{Swing}-biblioteket, er importeret i klassen \textit{Images}, giver enhver view-klasse adgang til at få fat i alle billeder. På samme måde er selv-definerede farver oprettet i klassen \textit{Colors}.